\documentclass[11pt, a4paper]{moderncv}
\moderncvstyle{banking} 
\moderncvcolor{blue}
\usepackage[utf8]{inputenc}
\usepackage[scale=0.85]{geometry}
\setlength{\hintscolumnwidth}{3cm} 
\usepackage{import}

\name{Junhao}{Zhang}
\title{Machine Learning Software Engineer} 
\phone[mobile]{+86 (0)132 6290 8770} 
\email{howiejayzh@gmail.com}
\social[github]{howiejayz}
\social[linkedin]{junhao-zh}
\homepage{www.howiejayz.com}


\begin{document}

\makecvtitle

\section{Professional Summary}
Innovative and results-driven Software Development Engineer specializing in machine learning and GPU programming. Proven expertise in optimizing performance for advanced computing systems, with a strong background in electronic engineering and applied machine learning. Adept at collaborating in high-tech environments to deliver cutting-edge solutions.

\section{Experience}
\cventry{Mar. 2024--Present}{Senior Software Development Engineer in Data Centre}{\includegraphics[height=1em]{logos/amd.jpeg} AMD}{Shanghai, China}{}{%
\begin{itemize}
    \item Optimising LLM training and inference, such as Llama2-7B and Llama2-70B 
    \item Supporting clients of AMD including Alibaba, Microsoft, and Kuaishou for their AI workload optimisations
\end{itemize}}

\cventry{Mar. 2022--Mar. 2024}{Software Development Engineer in Data Centre}{\includegraphics[height=1em]{logos/amd.jpeg} AMD}{Shanghai, China}{}{%
\begin{itemize}
    \item Participated in the development of \href{https://github.com/ROCm/flash-attention}{\textcolor{blue}{\underline{Flash Attention}}}, which is currently widely used by a variety of users, including Hugging Face, Microsoft, and individual AI developers in open-source communities.
    \item Spearheaded the optimization of machine learning algorithms through advanced GPU kernel programming and framework development.
    \item Focused on enhancing software capabilities for AMD GPUs, notably the MI210, MI250 and MI300.
    \item Contributed to our machine SDK, incorporating technologies like HIP (akin to CUDA) and Composable Kernel.
\end{itemize}}

\section{Education}
\cventry{Oct. 2020--Oct. 2021}{MSc. in Applied Machine Learning}{\includegraphics[height=1em]{logos/imperial.jpeg} Imperial College London}{London, UK}{\textit{Merit}}{}
\cventry{Sept. 2018--Jun. 2020}{BEng. in Electronic Engineering}{\includegraphics[height=1em]{logos/uol.jpeg} University of Liverpool}{Liverpool, UK}{\textit{First-Class Honours}}{}
\cventry{Sept. 2016--Jun. 2018}{BEng. in Electronic Science and Technology}{\includegraphics[height=1em]{logos/xjtlu.jpeg} Xi’an Jiaotong-Liverpool University}{Suzhou, China}{\textit{First-Class Honours}}{}

\section{Technical Skills}
\cvitem{Programming}{C++, CUDA, ROCm HIP, Python}
\cvitem{Libraries}{PyTorch, TensorFlow, scikit-learn, Numpy, Pandas}
\cvitem{Tools}{Git, Docker, Vim, Bash, Matlab, Latex}
\cvitem{Currently Learning}{Next.js, React, Typescript}

\section{Projects}
\cventry{Nov 2023--Present}{\href{https://github.com/ROCm/inference/tree/r4.0_sd_ort_v2}{\textcolor{blue}{\underline{Stable Diffusion for MLPerf v4.0 Inference Submission}}}}{\includegraphics[height=1em]{logos/amd.jpeg} AMD}{Shanghai, China}{}{
\begin{itemize}
    \item Optimized inference for Stable Diffusion XL for images at 1024x1024 on AMD MI300X GPUs.
    \item Tested different backends including PyTorch with xFormers and Onnxruntime for ROCm.
    \item Achieved inference speed at 0.40 images/s, which is close to the performance of NVIDIA A100
\end{itemize}
}

\cventry{May 2023--Present}{\href{https://github.com/ROCm/flash-attention}{\textcolor{blue}{\underline{Flash Attention For AMD ROCm}}}}{\includegraphics[height=1em]{logos/amd.jpeg} AMD}{Shanghai, China}{}{
\begin{itemize}
    \item Led the integration and optimization of the Flash Attention algorithm on AMD MI250 GPUs.
    \item Utilized PyTorch and Composable Kernel to significantly enhance computational efficiency and GPU utilization.
    \item Achieved a notable reduction in processing time, contributing to more efficient machine learning operations on GPUs.
    \item Maintaining the GitHub repository and solving issues from our customers and open-source community users.
\end{itemize}
}

\cventry{Jan. 2023--Apr. 2023}{\href{https://github.com/ROCm/inference/tree/junhzhan-r2.0-rnnt-gpu}{\textcolor{blue}{\underline{Enabling the GPU Inference for RNNT}}}}{\includegraphics[height=1em]{logos/amd.jpeg} AMD}{Shanghai, China}{}{
\begin{itemize}
    \item Oversaw the adaptation of the RNNT model for GPU-based inference, leveraging MLPerf v2.0 benchmarks.
    \item Achieved a tenfold increase in processing speed compared to traditional CPU-based methods.
    \item Enhanced the model's scalability and efficiency, setting a new standard in GPU-accelerated inferencing.
\end{itemize}
}

\cventry{Mar. 2022--Dec. 2022}{Investigation of DLRM Performance in HugeCTR and TorchRec}{\includegraphics[height=1em]{logos/amd.jpeg} AMD}{Shanghai, China}{}{
\begin{itemize}
    \item Conducted in-depth performance analysis of the DLRM benchmark, focusing on GPU efficiency.
    \item Compared and evaluated model throughput using TorchRec and PyTorch, identifying key areas for optimization.
    \item Provided actionable insights for performance improvements in GPU-based machine learning applications.
\end{itemize}
}

\cventry{Oct. 2020--Mar. 2021}{\href{https://github.com/howiejayz/Video-Player-Controlled-by-Action-Recognition}{\textcolor{blue}{\underline{EMG-Controlled Video Player}}}}{\includegraphics[height=1em]{logos/imperial.jpeg} Imperial College London}{London, UK}{}{
\begin{itemize}
    \item Innovated an EMG-based video player controlled through hand gestures, integrating advanced machine learning algorithms.
    \item Attained a remarkable 95\% accuracy in gesture prediction, greatly enhancing user interaction.
    \item Received Class A grade in the MSc program for the project's technical excellence and practical application.
\end{itemize}
}

\cventry{Oct. 2020 - Jan. 2021}{\href{https://github.com/howiejayz/SOMAS2020}{\textcolor{blue}{\underline{Multi-Agent System Using Go-Lang}}}}{\includegraphics[height=1em]{logos/imperial.jpeg} Imperial College London}{London, UK}{}{
\begin{itemize}
    \item Collaborated in a large-scale team project to develop a multi-agent system with intricate game theory dynamics.
    \item Designed and programmed a high-performance agent using Go-Lang, demonstrating superior strategy formulation.
    \item Achieved outstanding performance among competing agents, underscoring effective collaborative and competitive tactics.
\end{itemize}
}

\cventry{Nov. 2019 - Dec. 2019}{\href{https://github.com/howiejayz/graph-digitiser}{\textcolor{blue}{\underline{Graph Digitiser with Qt and C++}}}}{\includegraphics[height=1em]{logos/uol.jpeg} University of Liverpool}{Liverpool, UK}{}{
\begin{itemize}
    \item Engineered a sophisticated graph digitiser tool in C++ and Qt, catering to a wide range of data visualization needs.
    \item Enabled accurate digitization of complex image data, including 2D maps and mathematical plots.
    \item The tool was lauded for its user-friendly interface and precision, earning an 80\% grade for functionality and innovation.
\end{itemize}
}


\section{Interests}
\cvitem{}{Hiking, Scuba diving, Photography}

\end{document}
